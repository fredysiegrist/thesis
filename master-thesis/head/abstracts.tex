%\begingroup
%\let\cleardoublepage\clearpage


% English abstract
\cleardoublepage
\chapter*{Abstract}
%\markboth{Abstract}{Abstract}
\addcontentsline{toc}{chapter}{Abstract (English / Deutsch)} % adds an entry to the table of contents

The basic idea of this project was to implement codon bias indicies that have been listed and mathematically described before \cite{BinderCh13} together with important indices that came after and improvements. 
The standard genetic code, as we know it for most of the species, has 61 sense codons translating into 20 amino acids. These codons are usually decoded by a much larger number of different tRNAs, that in addition can ‘wobble’, that is some tRNAs are used to read more than one codon. For example, yeast has 41 tRNAs to decode its 61 sense codons. The choice of synonymous codon is not at all random, some codons are used frequently and others are hardly used at all \cite{Cannarozzi2010}. In this project, we implemented several functions that compute indices of codon usage and applied them to coding sequences of the grass plant \textit{Eragostis tef}. We set up an R project based on another package seqinr and transcribed the code for standard indicies that measure codon usage and bias such as the codon adaptation index (CAI), the Number of effective codons (Nc) and others from a code-library called CodonIndices of the language Darwin.\\
Most importantly, functions were written for text- and graphical-based visualization of analyses based on codon usage that can be asses by demonstrations of the package. Further work may assess the significance of groups with extreme values of these indices using Gene-Ontology Enrichment analyses.\\
The package compromises of a set of functions to handle genomic data, to compute some codon indices, as well as tests, demonstrations, sample data and documentation thereof and provides a backbone for further implementation of a complete codon bias analysis instrument.


\vskip0.5cm
Key words: 

R package; Codon Bias; Codon Usage; Statistical Analysis


% German abstract

\cleardoublepage
\chapter*{Zusammenfassung}
%\markboth{Zusammenfassung}{Zusammenfassung}
Das Grundprinzip dieses Projektes war das Implementieren von Kodon-Vorliebe Indexen die zuvor schon mathematisch aufgelistet wurden \cite{BinderCh13} sowie weitere Indexe die später dazu kamen und Verbesserungen in den schon Bestehenden. 
Der genetische Standardcode, wie wir ihn in den meisten Spezies antreffen, hat 61 kodierende Trinukleotide die in 20 Aminosäuren übersetzt werden. Diese Kodons werden üblicherweise von einer viel grösseren Anzahl an tRNAs gelesen, die zusätzlich auch noch ‘wobbeln’ können, das bedeutet, dass einige tRNAs benützt werden können um mehr als ein unterschiedliches Kodon zu lesen. Zum Beispiel reichen in der Hefe 41 tRNA um alle 61 Kodons zu dekodieren. Die Wahl der synonymen Kodons ist in der Tat alles andere als zufällig und einige Kodons werden weit öfters verwendet als Andere, welche zuweilen fast gar nicht benützt werden \cite{Cannarozzi2010}. In diesem Projekt haben wir verschiedenste Funktionen eingebettet, die die Indexe für Kodon-Vorliebe ausrechnen und haben diese zur Berechnung von Gensequenzen der Graspflanze \textit{Eragostis tef} eingesetzt. Wir haben ein R Projekt entwickelt, basiert auf einem Fremdpaket seqinr und haben den Kode für Standardindexe umgeschrieben welche die Kodon-Vorliebe ausrechnen wie zum Beispiel den Kodonadaptionsindex (CAI), die Anzahl effektiver Kodons (Nc) und Weitere von einer Kodesammlung frmit dem Namen CodonIndices der Programmiersprache Darwin.\\
Hauptsächlich wurden Funktionen geschrieben um die Text- und Bild-basierte Visualisierung der Analysen basierend auf der Kodon-Vorliebe darzustellen, welche durch die Demonstrationsfunktionen des Pakets aufgerufen werden können. Weiterführende Arbeiten können dann die Signifikanz von Gruppen mit extremen Werten für diese Indexe in genontologischen Anreicherungsanalysen untersuchen.\\
Das vorliegende Paket beinhaltet ein Satz an Funktionen zum Bearbeiten der genomische Daten, um einige Kodonindexe auszurechnen, sowie Test-, Demonstrationsfunktionen und Beispieldaten sowie eine Dokumentationen davon und bildet ein Skelett für weitere Einbindungen um ein vollständiges Kodon-Vorliebe-Analyse-Instrument zu erstellen.
\vskip0.5cm
Stichwörter: 

R Sammlung; Kodon-Vorliebe; Kodon-Gebrauch; Statistische Analyse




% French abstract

%\cleardoublepage
%\chapter*{Résumé}
%\markboth{Résumé}{Résumé}
% put your text here
%\lipsum[1-2]
%\vskip0.5cm
%Mots clefs: 
%put your text here



%\endgroup			
%\vfill
